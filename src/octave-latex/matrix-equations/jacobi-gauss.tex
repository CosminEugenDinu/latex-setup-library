\documentclass[12pt]{article}
\usepackage[a5paper,margin=7mm,bottom=15mm]{geometry}
\usepackage{amsmath}

\title{Tema 2 - Calcul numeric - an I ID}
\date{Due May 15, 2021 10:00 AM}
\author{Dinu Eugen Cosmin}

\begin{document}
\maketitle
\thispagestyle{empty}

% Proper formatting of paragraph with sloppypar
\begin{sloppypar}
1. Se da matricea 
$A=
\begin{bmatrix}
    3 & -1 & -1 \\ 
    -1 & 3 & -1  \\ 
    1 & -1 & 4 
\end{bmatrix}$.

a) Determinati daca matricea este diagonal dominanta.

b) Rezolvati sistemul $Ax=b$ unde $b =
\begin{bmatrix}
   4 \\
   0 \\
   3 
\end{bmatrix}$ folosig metoda Jacobi cu $x^{(0)}=0$ pentru $\varepsilon=0.05$.

c) Rezolvati sistemul $Ax=b$ unde $b=
\begin{matrix}
    4 \\
    0 \\
    3 \\
\end{matrix}$ folosind metoda Gauss-Siedel cu $x^{(0)}=0$ pentru $\varepsilon=0.05$.

\end{sloppypar}
\pagebreak


\begin{sloppypar}
(a) Matricea este diagonal dominantă deoarece îndeplinește condiția: 
${\displaystyle |a_{ii}|\geq \sum _{j\neq i}|a_{ij}|\quad}}$.\\

Cod octave:
\textbf{isDiagonallyDominant.m}
\begin{scriptsize}\begin{verbatim}
    function isDD = isDiagonallyDominant(matrix)
        for i=1:size(matrix,1)
          a = abs(matrix(i,i));
          sum = 0;
          for j=1:size(matrix,2)
            if i != j
              sum += abs(matrix(i,j));
            endif
          endfor
          if sum > a
            isDD = false;
            return
          else
            sum = 0;
            continue
          endif
        endfor
        isDD = true;
    endfunction
\end{verbatim}\end{scriptsize}

Rezultat:
\begin{scriptsize}\begin{verbatim}
    >> A = [3 -1 -1; -1 3 -1; 1 -1 4];
    >> isDiagonallyDominant(A)
    ans = 1
\end{verbatim}\end{scriptsize}
\newpage

(b) Cod octave metoda Jacobi: \textbf{jacobi.m}
\begin{scriptsize}\begin{verbatim}
    function x = jacobi(A,b,epsilon)
      [n, m] = size(A);
      x = rand(n,1);
      xnew = zeros(A,1);
      iter = 0;
      d = 0;
      diff=zeros(A,1);
      while d<n && iter<100000
        for i=1:n
          S=0;
          for j=1:n
            if i != j
              S=S+A(i,j)*x(j);
            endif
          endfor
          xnew(i)=(b(i)-S)/A(i,i);
        endfor
        for i=1:n
          diff(i)=abs(xnew(i)-x(i));
          if diff(i) <= epsilon
            d++;
          endif
        endfor
        x = xnew;
        iter ++;
      endwhile
      if d < n
        printf("nr de iteratii depasit\n");
      endif
    endfunction
    
    A = [3 -1 -1; -1 3 -1; 1 -1 4];
    b = [4; 0; 3];
\end{verbatim}\end{scriptsize}
Rezultat:
\begin{scriptsize}\begin{verbatim}
    jacobi(A,b,0.05)
    ans =
       1.7365
       0.7398
       0.4975

       linsolve(A,b)
    ans =
       1.7500
       0.7500
       0.5000
\end{verbatim}\end{scriptsize}
\newpage

(c) Cod octave metoda Gauss-Siedel: \textbf{gauss.m}
\begin{scriptsize}\begin{verbatim}
    function x = gauss(A,b,epsilon)
        [m, n] = size(A);
        x = zeros(n,1);
        normVal=Inf; 
        while normVal>epsilon
          x_old=x;
          for i=1:n
            sigma=0;
            for j=1:i-1
              sigma=sigma+A(i,j)*x(j);
            endfor 
            for j=i+1:n
              sigma=sigma+A(i,j)*x_old(j);
            endfor
            x(i)=(1/A(i,i))*(b(i)-sigma);
          endfor
          normVal=norm(x_old-x);
        endwhile
    endfunction
\end{verbatim}\end{scriptsize}
Rezultat:
\begin{scriptsize}
    \begin{verbatim}
    A = [3 -1 -1; -1 3 -1; 1 -1 4];
    b = [4; 0; 3];
    >> gauss(A,b,0.05)
    ans =
       1.7524
       0.7514
       0.4997

    >> linsolve(A,b)
    ans =
       1.7500
       0.7500
       0.5000
    \end{verbatim}
\end{scriptsize}
\end{sloppypar}
\end{document}